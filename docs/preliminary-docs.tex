
\documentclass[12pt]{article}
\usepackage[T1]{fontenc}
\usepackage[polish]{babel}
\usepackage{hyperref}
\usepackage[ bottom=30mm, top=20mm, left=30mm, right=30mm]{geometry}

\title{Przeszukiwanie i optymalizacja - dokumentacja wstępna}
\author{Maksym Bieńkowski, Jędrzej Grabski}
\date{01.12.2024}

\begin{document}
\maketitle
\begin{centering}
	\textbf{Temat projektu: }Algorytm roju cząstek z modyfikacjami dotyczącymi współczynnika bezwładności
\end{centering}

\section{Analiza problemu}

Ideą Algorytmu Roju Cząstek (PSO - Particle Swarm Optimalization), jest symulowanie populacji ("roju"),
która rozwija się na podstawie wiedzy pojedynczych osobników ("cząstek") oraz pewnej wiedzy dzielonej.
Każda z cząstek posiada swoją prędkość oraz pozycję w przestrzeni rozwiązań.
Ponadto zapamiętywane jest najlepsze rozwiązanie znalezione do tej pory przez każdą z cząstek (optimum lokalne, niewspółdzielone
z resztą populacji), a także najlepsze rozwiązanie z całego roju (optimum globalne, współdzielone przez wszystkie cząstki).

Prędkość \(i\)-tej cząstki w epoce \(k+1\) dana jest następującym wzorem:\[V_i(k+1) = wV_i(k) + \phi_p r_1(P_i(k) - X_i(k)) + \phi_g r_2(P_g(k) - X_i(k))\]
gdzie \(w\) oznacza współczynnik bezwładności, \(P_i\) położenie optimum lokalnego, \(P_g\) położenie optimum globalnego, współczynniki \(r_1\) i \(r_2\)
losowane są z rozkładem \(U[0, 1]\), a \(\phi_p\) oraz \(\phi_g\) oznaczają kolejno parametryzowane współczynniki wagi - poznawczy i społeczny. Na podstawie powyższego wzoru obserwujemy,
że wektor prędkości tworzony jest na podstawie trzech składowych, a współczynnik bezwładności określa wagę składowej będącej
prędkością w poprzedniej iteracji. Im mniejsza wartość tego współczynnika, tym bardziej zwrotne i skłonne do eksploatacji są cząstki.
Ze zwiększeniem wartości współczynnika bezwładności wiąże się natomiast większa skłonność do eksploracji przestrzeni, co może jednak skutkować "przestrzeliwaniem"\space
optimów lokalnych.



\section{Propozycja rozwiązania}

Problem sformułowany w poprzedniej sekcji spróbujemy zniwelować poprzez wprowadzenie dynamicznej zmiany współczynnika bezwładności,
uzależniając go od liczby wykonanych iteracji. Współczynnik będzie stopniowo zmniejszany się w miarę pracy algorytmu.
Umożliwi to skupienie się na eksploracji w początkowej fazie algorytu, a następnie bardziej precyzyjne
zbieganie wokół optimów pod koniec pracy.

\section{Przyjęte założenia}

\subsection*{Przestrzeń rozwiązań}
Zakładamy, że przestrzeń rozwiązań jest ciągła, ograniczona i wielowymiarowa, a wartości funkcji celu są dobrze zdefiniowane w całej przestrzeni.

\subsection*{Funkcja celu}
Funkcja celu jest różnorodna, tj. może być jedno- lub wielomodalna, aby przetestować algorytm w różnych warunkach.

\subsection*{Początkowa populacja}
Pozycje cząstek w roju są inicjalizowane losowo z rozkładem jednostajnym w granicach przestrzeni rozwiązań. Początkowe prędkości cząstek będą losowane z rozkładem jednostajnym na bazie ograniczeń przestrzeni przeszukiwań.

\subsection*{Parametry algorytmu}
\begin{itemize}
	\item{Liczba cząstek w roju oraz liczba iteracji są ustalane na początku i pozostają stałe w trakcie pracy algorytmu.}
	\item{ Parametry algorytmu dobrane zostaną na podstawie literatury. }
	\item{Dynamiczny współczynnik bezwładności: Zmiana współczynnika bezwładności następuje zgodnie z wzorem $w'=w*u^{-k}$, gdzie $w \in [0,1]$ to bazowy współczynnik bezwładności, $u \in [1.0001, 1.005]$ to siła wytracania wartości współczynnika, a $k$ to numer iteracji.}
\end{itemize}

\section{Sposób badania jakości rozwiązania}

\subsection*{Porównanie algorytmów}
Algorytm PSO z dynamicznym współczynnikiem bezwładności zostanie porównany z klasyczną wersją tego algorytmu (ze stałym współczynnikiem).
Testy zostaną przeprowadzone na dobrze znanych benchmarkach funkcji optymalizacyjnych w zróżnicowanych ilościach wymiarów.

\subsection*{Kryteria oceny}
\begin{itemize}
	\item{Jakość rozwiązania: Różnica wartości funkcji celu algorytmu od optimum globalnego dla danego problemu optymalizacji po określonej liczbie iteracji.}

	\item{Szybkość zbieżności: Analiza tempa zbilżania się do optimum, mierzona przez ilość iteracji po której zostało osiągnięte otoczenie $\epsilon$ optimum globalnego.}
\end{itemize}

\subsection*{Proces badawczy}
Testy zostaną przeprowadzone na przestrzeniach o różnych wymiarach, aby ocenić wpływ wymiarowości na działanie algorytmu oraz
efektywność wprowadzonej modyfikacji.

\subsection*{Wizualizacja}
Dla każdego przeprowadzonego badania przedstawiony będzie wykres wartości funkcji celu w \(P_g\) na przestrzeni iteracji.
Ponadto zamierzamy zwizualizować trajektorię populacji w niskowymiarowej przestrzeni na wykresie poziomicowym.

\section{Środowisko}
\begin{itemize}
	\item{Projekt będzie realizowany w środowisku Python.}
	\item{Narzędziem do zarządzania zależnościami oraz środowiskiem wirtualnym będzie \href{https://pdm-project.org/latest/PDM}{\textit{PDM}}.}
	\item{Do wizualizacji przebiegów działań algorytmów zostanie użyta biblioteka \textit{matplotlib} lub \href{https://seaborn.pydata.org/}{\textit{seaborn}}}


\end{itemize}
\section{Bibliografia}
\begin{itemize}
	\item{Particle swarm optimization. (1995). IEEE Conference Publication | IEEE Xplore. https://ieeexplore.ieee.org/document/488968}
	\item{Jiao, B., Lian, Z., \& Gu, X. (2006). A dynamic inertia weight particle swarm optimization algorithm. Chaos Solitons \& Fractals, 37(3), 698–705. https://doi.org/ 10.1016/j.chaos.2006.09.063}
\end{itemize}
\end{document}
