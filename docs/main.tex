\documentclass[12pt]{article}
\usepackage[T1]{fontenc}
\usepackage[polish]{babel}
\usepackage{hyperref}
\usepackage[ bottom=30mm, top=20mm, left=25mm, right=25mm]{geometry}
\usepackage{placeins}
\usepackage{graphicx}
\usepackage{subcaption}
\usepackage{amsmath}
\graphicspath{{../plots/used/}}

\title{Przeszukiwanie i optymalizacja - dokumentacja końcowa}
\author{Maksym Bieńkowski, Jędrzej Grabski}
\date{29.01.2025}

\begin{document}
\maketitle
\begin{centering}
	\textbf{Temat projektu: }Algorytm roju cząstek z modyfikacjami dotyczącymi współczynnika bezwładności
\end{centering}

\section{Analiza problemu}

Ideą Algorytmu Roju Cząstek (PSO - Particle Swarm Optimalization), jest symulowanie populacji ("roju"),
która rozwija się na podstawie wiedzy pojedynczych osobników ("cząstek") oraz pewnej wiedzy dzielonej.
Każda z cząstek posiada swoją prędkość oraz pozycję w przestrzeni rozwiązań.
Ponadto zapamiętywane jest najlepsze rozwiązanie znalezione do tej pory przez każdą z cząstek (optimum lokalne, niewspółdzielone
z resztą populacji), a także najlepsze rozwiązanie z całego roju (optimum globalne, współdzielone przez wszystkie cząstki).

Prędkość \(i\)-tej cząstki w epoce \(k+1\) dana jest następującym wzorem:\[V_i(k+1) = wV_i(k) + \phi_p r_1(P_i(k) - X_i(k)) + \phi_g r_2(P_g(k) - X_i(k))\]
gdzie \(w\) oznacza współczynnik bezwładności, \(P_i\) położenie optimum lokalnego, \(P_g\) położenie optimum globalnego, współczynniki \(r_1\) i \(r_2\)
losowane są z rozkładem \(U[0, 1]\), a \(\phi_p\) oraz \(\phi_g\) oznaczają kolejno parametryzowane współczynniki wagi - poznawczy i społeczny. Na podstawie powyższego wzoru obserwujemy,
że wektor prędkości tworzony jest na podstawie trzech składowych, a współczynnik bezwładności określa wagę składowej będącej
prędkością w poprzedniej iteracji. Im mniejsza wartość tego współczynnika, tym bardziej zwrotne i skłonne do eksploatacji są cząstki.
Ze zwiększeniem wartości współczynnika bezwładności wiąże się natomiast większa skłonność do eksploracji przestrzeni, co może jednak skutkować "przestrzeliwaniem"\space
optimów lokalnych.



\section{Badane rozwiązanie}

Problem sformułowany w poprzedniej sekcji spróbujemy zniwelować poprzez wprowadzenie dynamicznej zmiany współczynnika bezwładności,
uzależniając go od liczby wykonanych iteracji. Współczynnik będzie stopniowo zmniejszany się w miarę pracy algorytmu.
Umożliwi to skupienie się na eksploracji w początkowej fazie algorytmu, a następnie bardziej precyzyjne
zbieganie wokół optimów pod koniec pracy. W iteracji \(k\) wartość współczynnika bezwładności opisana jest wzorem
\[w_{k} = w_{k-1}u^{-k}\]
gdzie  \(u \in [1.001, 1.005]\) na podstawie literatury.

\begin{center}
	\rule{0.8\linewidth}{0.4pt}
\end{center}

Po konsultacji z prowadzącym, proponujemy dwa dodatkowe sposoby modyfikacji czynnika bezwładności.
\begin{itemize}
	\item Liniowe zmniejszanie wraz z liczbą iteracji - co iterację wartość współczynnika zmniejszana
	      jest o stałą wartość równą $\frac{w_{max} - w_{min}}{n}$, gdzie n oznacza całkowitą liczbę iteracji
	      przydzieloną algorytmowi, a $w_{max}$ i $w_{min}$ oznaczają kolejno startową i końcową wartość współczynnika.
	\item Adaptacyjna zmiana współczynnika - jeśli w danej iteracji wartość globalnego optimum została zaktualizowana, wartość współczynnika
	      jest odpowiednio zmniejszana, aby wspierać eksploatację znalezionego rozwiązania. W przeciwnym wypadku jest ona podwyższana:
	      \[
		      w_{k+1} =
		      \begin{cases}
			      1.01w_{k}, & \text{jeśli } gbest_{k+1} = gbest_{k}    \\
			      0.95w_{k}, & \text{jeśli } gbest_{k+1} \neq gbest_{k}
		      \end{cases}
	      \]
	      Wartości $1.01$ oraz $0.95$ zostały dobrane eksperymentalnie.
\end{itemize}

\section{Sposób przeprowadzania badań, przyjęte założenia}

Badania przeprowadzone zostały w ciągłej, ograniczonej wielowymiarowej przestrzeni z dobrze zdefiniowanymi wartościami funkcji celu w każdym punkcie.
Zbadane zostało działanie algorytmu zarówno dla funkcji jedno-, jak i wielomodalnych.Ograniczenia przestrzeni zostały zrealizowane w wariancie lamarkowskim,
przy pomocy rzutowania na ograniczenia w każdym wymiarze ograniczeń kostkowych. Ustaliliśmy klasyczne ograniczenia $[-2.048, 2.048]$ dla każdego wymiaru.


Algorytmy badaliśmy na następujących funkcjach:

\begin{itemize}
	\item \textbf{Funkcja sferyczna}  \[y(x) = \sum_{i = 0}^{d} x_i^2 \] unimodalna funkcja kwadratowa, trywialne
	      zadanie optymalizacji, minimum globalne $f(x^*) = 0$ dla $x^*=0$.

	\item \textbf{Funkcja Rastrigina} \[f(x) = A d + \sum_{i=1}^{d} \left( x_i^2 - A \cos(2\pi x_i) \right)\]
	      klasyczna wielomodalna
	      funkcja stosowana do ewaluacji algorytmów optymalizacyjnych posiadająca wiele minimów lokalnych.
	      Minimum globalne $f(x^*) = 0$ dla $x^*=0$.
	      Zastosowaliśmy klasyczny wariant z \(A = 10\).

	\item \textbf{Funkcja Ackleya} \[f(x) = -a \exp \left( -b \sqrt{\frac{1}{d} \sum_{i=1}^{d} x_i^2} \right)
		      - \exp \left( \frac{1}{d} \sum_{i=1}^{d} \cos(c x_i) \right) + a + \exp(1)\]
	      Kolejna znana wielomodalna funkcja benchmarkowa, przyjęliśmy \(a = 20, b = 0.2, c = 2\pi\). Minimum globalne
	      $f(x^*) = 0$ dla $x^*=0$.

	\item \textbf{Funkcja Rosenbrocka} \[f(x) = \sum_{i=1}^{d-1} \left[ 100 (x_{i+1} - x_i^2)^2 + (x_i - 1)^2 \right]\]
	      Liczba optimów funkcji zależy od wymiarowości - dla \(d <= 3\) posiada ona jedno minimum globalne $f(x^*) = 0$ dla $x^* = 1$.
	      Charakteryzuje się płaską, wykrzywioną doliną, dotarcie do optimum wymaga precyzyjnej nawigacji i bywa kosztowne czasowo.
\end{itemize}

W powyższych równaniach \(d\) oznacza rozmiar przestrzeni. Wyniki były uśredniane z 50 uruchomień dla wymiarowości \(d \in \{5, 10, 20\}\).
Rozwiązanie było oceniane względem dwóch kryteriów - wartości f. celu w znalezionym punkcie oraz szybkości
zbieżności algorytmów - iteracji wymaganych do odnalezienia tego punktu.

\subsection{Parametry algorytmu}
Ze względu na dużą liczbę parametrów, część z nich musiała zostać ustalona, aby posiadany przez nas sprzęt sprostał wymaganiom czasowym:

\begin{itemize}
	\item We wszystkich przypadkach populacja liczyła 50 osobników.
	\item Przy każdym uruchomieniu $\phi_p = \phi_g = 2.0$.
	\item Początkowe pozycje cząstek generowane były zgodnie z rozkładem jednostajnym w granicach przestrzeni rozwiązań.
	\item Początkowa prędkość cząstek była generowana z rozkładem jednostajnym w zakresie $[-\frac{x_{max, j}-x_{min, j}}{2}, \frac{x_{max, j}-x_{min, j}}{2}]$ dla każdego wymiaru \(j\), gdzie $x_{max}$ i $x_{min}$ oznaczają odpowiednio
	      górne i dolne ograniczenie w tym wymiarze.
	\item Za każdym razem algorytmy były uruchamiane z maksymalną liczbą 300 iteracji oraz $eps = 1e-5$. Algorytm zapamiętuje,
	      w której iteracji dotarł do epsilona, co pozwala zmierzyć zbieżność do punktu o określonej jakości. W praktyce tak niska
	      wartość okazała się oczywiście zbyt niska dla wielowymiarowych funkcji i wymagałaby indywidualnego arbitralnego strojenia
	      w zależności od funkcji i wymiaru, brana była pod uwagę przy badaniu \ref{convergence_table}.
	\item Wartość współczynnika bezwładności lub, w przypadku dynamicznego współczynnika bezwładności $w = 0.1$
	\item Wartość współczynnika $u$ używanego do obniżania wartości bezwładności $u = 1.0001$
\end{itemize}

Początkowa wartość współczynnika bezwładności została wybrana eksperymentalnie przez zbadanie najlepszej wartości f. Rosenbrocka w 10 wymiarach w zależności od
jego wartości. Wyniki znajdują się w tabeli \ref{tab:test_rosenbrock}.

\begin{table}[h]
	\centering
	\begin{tabular}{|c|c|c|c|c|c|}
		\hline
		DI / $\omega_0$ & 0.05 & 0.1 & 0.2 & 0.4 & 0.7  \\
		\hline
		Nie             & 36   & 34  & 47  & 62  & 1798 \\
		Tak             & 30   & 29  & 40  & 60  & 1539 \\
		\hline
	\end{tabular}
	\caption{Zaokrąglona wartość f. Rosenbrocka dla najlepszego znalezionego punktu w zależności od startowego współczynnika bezwładności}
	\label{tab:test_rosenbrock}
\end{table}


Badanie było przeprowadzone dla różnych funkcji, wyniki były zbliżone.
Analogiczna próba została przeprowadzona w celu ustalenia optymalnej wartości współczynnika $u$.


\pagebreak
\section{Wyniki badań}
\subsection*{Porównanie ze względu na jakość rozwiązań}

\begin{table}[ht]
	\centering
	\begin{tabular}{|c|c|c|c|}
		\hline
		DI  / Dim & 5             & 10     & 20     \\
		\hline
		Nie       & $8.90e^{-12}$ & 0.0184 & 2.2144 \\
		Tak       & $6.93e^{-12}$ & 0.0243 & 2.2864 \\
		\hline
	\end{tabular}
	\caption{Wartość f. sferycznej dla najlepszego znalezionego punktu w zależności od wymiaru}
	\label{tab:sphere_values}
\end{table}

\begin{table}[ht]
	\centering
	\begin{tabular}{|c|c|c|c|}
		\hline
		DI / Dim & 5     & 10     & 20      \\
		\hline
		Nie      & 3.546 & 29.157 & 413.303 \\
		Tak      & 2.516 & 33.790 & 414.403 \\
		\hline
	\end{tabular}
	\caption{Wartość f. Rosenbrocka dla najlepszego znalezionego punktu w zależności od wymiaru}
	\label{tab:rosenbrock_values}
\end{table}

\begin{table}[ht]
	\centering
	\begin{tabular}{|c|c|c|c|}
		\hline
		DI / Dim & 5     & 10     & 20     \\
		\hline
		Nie      & 7.959 & 18.453 & 44.838 \\
		Tak      & 7.697 & 19.518 & 46.823 \\
		\hline
	\end{tabular}
	\caption{Wartość f. Rastrigina dla najlepszego znalezionego punktu w zależności od wymiaru}
	\label{tab:rastrigin_values}
\end{table}

\begin{table}[ht]
	\centering
	\begin{tabular}{|c|c|c|c|}
		\hline
		DI / Dim & 5     & 10    & 20    \\
		\hline
		Nie      & 0.346 & 1.486 & 3.102 \\
		Tak      & 0.402 & 1.578 & 3.038 \\
		\hline
	\end{tabular}
	\caption{Wartość f. Ackleya dla najlepszego znalezionego punktu w zależności od wymiaru}
	\label{tab:ackley_values}
\end{table}

\pagebreak
\subsection*{Porównanie ze względu na zbieżność}

\begin{table}[h]
	\centering
	\begin{tabular}{|c|c|c|c|c|}
		\hline
		DI / Funkcja & Sferyczna & Ackley & Rosenbrock & Rastrigin \\
		\hline
		Nie          & 37        & 57     & NA         & NA        \\
		Tak          & 27        & 243    & NA         & NA        \\
		\hline
	\end{tabular}
	\caption{Uśrednione iteracje do znalezienia $eps$, 2 wymiary}
	\label{convergence_table}
\end{table}

\FloatBarrier

\subsection*{Zmiana najlepszego punktu na przestrzeni iteracji dla wybranych funkcji}

\begin{figure}[ht]
	\centering
	\begin{subfigure}{0.49\textwidth}
		\includegraphics[width=\textwidth]{global_best_Ackley_10d_10_True.png}
		\caption{Tempo zbieżności algorytmu z dynamiczną inercją - f. Ackleya, 10 wymiarów}
	\end{subfigure}
	\hfill
	\begin{subfigure}{0.49\textwidth}
		\includegraphics[width=\textwidth]{global_best_Ackley_10d_10_False.png}
		\caption{Tempo zbieżności algorytmu ze statyczną inercją - f. Ackleya, 10 wymiarów}
	\end{subfigure}
	\caption{Porównanie tempa zbieżności algorytmu dla dynamicznej i statycznej inercji w 20 wymiarach dla 300 iteracji}
	\label{20d_comp}
\end{figure}


\begin{figure}[ht]
	\centering
	\begin{subfigure}{0.49\textwidth}
		\includegraphics[width=\textwidth]{global_best_Rastrigin_4d_4_True.png}
		\caption{Tempo zbieżności algorytmu z dynamiczną inercją - f. Rastrigina, 4 wymiary}
	\end{subfigure}
	\hfill
	\begin{subfigure}{0.49\textwidth}
		\includegraphics[width=\textwidth]{global_best_Rastrigin_4d_4_False.png}
		\caption{Tempo zbieżności algorytmu ze statyczną inercją - f. Rastrigina, 4 wymiary}
	\end{subfigure}
	\caption{Porównanie tempa zbieżności algorytmu dla dynamicznej i statycznej inercji w 4 wymiarach dla 300 iteracji}
	\label{4d_comp}
\end{figure}

\begin{figure}[ht]
	\centering
	\begin{subfigure}{0.49\textwidth}
		\includegraphics[width=\textwidth]{global_best_Sphere_2d_2_True.png}
		\caption{Tempo zbieżności algorytmu z dynamiczną inercją - f. sferyczna, 2 wymiary}
	\end{subfigure}
	\hfill
	\begin{subfigure}{0.49\textwidth}
		\includegraphics[width=\textwidth]{global_best_Sphere_2d_2_False.png}
		\caption{Tempo zbieżności algorytmu ze statyczną inercją - f. sferyczna, 2 wymiary}
	\end{subfigure}
	\caption{Porównanie tempa zbieżności algorytmu dla dynamicznej i statycznej inercji w 2 wymiarach dla 100 iteracji}
	\label{2d_comp}
\end{figure}


\begin{figure}[ht]
	\centering
	\begin{subfigure}{0.49\textwidth}
		\includegraphics[width=\textwidth]{global_best_Rastrigin_3d_3_True.png}
		\caption{Tempo zbieżności algorytmu z dynamiczną inercją - f. Rastrigina, 3 wymiary}
	\end{subfigure}
	\hfill
	\begin{subfigure}{0.49\textwidth}
		\includegraphics[width=\textwidth]{global_best_Rastrigin_3d_3_False.png}
		\caption{Tempo zbieżności algorytmu ze statyczną inercją - f. Rastrigina, 3 wymiary}
	\end{subfigure}
	\caption{Porównanie tempa zbieżności algorytmu dla dynamicznej i statycznej inercji w 3 wymiarach dla 300 iteracji}
	\label{3d_comp}
\end{figure}

\pagebreak
\FloatBarrier


\subsection*{Wizualizacja punktów przed i po działaniu algorytmu}

\begin{figure}[h!]
	\centering
	\begin{subfigure}{0.49\textwidth}
		\includegraphics[width=\textwidth]{start_end_Ackley_2d_2_True.png}
		\caption{Uśrednione punkty startowe i końcowe - f. Ackleya, 2 wymiary, dynamiczna inercja}
	\end{subfigure}
	\hfill
	\begin{subfigure}{0.49\textwidth}
		\includegraphics[width=\textwidth]{start_end_Ackley_2d_2_False.png}
		\caption{Tempo zbieżności algorytmu ze statyczną inercją - f. Rastrigina, 2 wymiary}
	\end{subfigure}
	\caption{Uśrednione punkty startowe i końcowe - f. Ackleya, 2 wymiary, statyczna inercja}
	\label{ackley_start_end}
\end{figure}

\subsection*{Wnioski}
Na podstawie tabel \ref{tab:sphere_values}, \ref{tab:rosenbrock_values}, \ref{tab:rastrigin_values}, \ref{tab:ackley_values} możemy stwierdzić, że
zastosowanie dynamicznego współczynnika bezwładności nie wydaje się negatywnie wpływać na jakość znajdowanych rozwiązań.

Wykresy \ref{20d_comp} oraz \ref{4d_comp} wydają się pokazywać, że dla nastaw, które wydawały się optymalne, w większych wymiarowościach nie jest widoczne tak duże
przyspieszenie zbieżności, jak możnaby się spodziewać. Nie wykluczamy, że jest to spowodowane konkretnymi przyjętymi parametrami algorytmu. Przeprowadziliśmy
także badania dla mniejszej liczby iteracji, jednak wydaje się, że w większej wymiarowości poprawa nie jest odczuwalna. W analizowanej literaturze przyspieszenie
wystąpiło wyłącznie dla niektórych funkcji benchmarkowych - być może obierając inne przypadki wyniki byłyby bardziej obiecujące.

Przy doborze innych parametrów udało się nam zauważyć korzyści wynikające z zastosowania dynamicznego współczynnika bezwładności, w szczególności dla:
\begin{itemize}
	\item wysokiej bezwładności początkowej
	\item niższego wymiaru problemu
	\item niższej liczby możliwych iteracji
	\item mniejszej populacji
\end{itemize}

Wobec tego, jeśli obliczenie funkcji celu jest kosztowne i możemy sobie pozwolić wyłącznie na niższą liczbę iteracji, istnieje korzyść z zastosowania tego rozwiązania.
Z pewnością nie wydaje się ono degenerować jakości wyników algorytmu, nawet w wyższych wymiarach, a aktualizacja wagi jest praktycznie darmowa w kwestii wymaganych obliczeń.
Ponadto, zgodnie z wykresem \ref{2d_comp}, dzięki gradualnemu obniżaniu bezwładności, algorytm ma mniejszą tendencję do przypadkowego błądzenia po przestrzeni rozwiązań i
dokładniej eksploatuje znalezione minima. Porównanie \ref{3d_comp} pokazuje, że wersja z dynamiczną bezwładnością wydaje się mieć również tendencje do znajdywania podobnych
rozwiązań dla gorszych punktów startowych. W wizualizacji \ref{ackley_start_end} obserwujemy, że przy zastosowaniu dynamicznej inercji punkty końcowe są gęściej skoncentrowane
w optimum niż w przypadku bez inercji. Jest to spowodowane zwiększoną eksploatacją w końcowych fazach algorytmu. Dla niższej wymiarowości i odpowiednich parametrów, algorytm znajdował
odpowiednio dobre punkty szybciej, co zobrazowane zostało w tabeli \ref{convergence_table}.

W kwestii porównania zachowań algorytmu w zależności od funkcji, najtrudniejszym zadaniem wydaje się funkcja Rastrigina. Dla wyższych wymiarowości niż 4 nie znajdował on
konsekwentnie optimum globalnego i cechował się tendencją do utykania w napotkanych optimach lokalnych, co także obrazuje porównanie \ref{3d_comp}.

\section{Bibliografia}
\begin{itemize}
	\item{Particle swarm optimization. (1995). IEEE Conference Publication | IEEE Xplore. https://ieeexplore.ieee.org/document/488968}
	\item{Jiao, B., Lian, Z., \& Gu, X. (2006). A dynamic inertia weight particle swarm optimization algorithm. Chaos Solitons \& Fractals, 37(3), 698–705. https://doi.org/ 10.1016/j.chaos.2006.09.063}
\end{itemize}
\end{document}
