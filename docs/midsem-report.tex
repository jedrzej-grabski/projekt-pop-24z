\documentclass[12pt]{article}
\usepackage[T1]{fontenc}
\usepackage[polish]{babel}
\usepackage{hyperref}
\usepackage[ margin=20mm]{geometry}

\title{Przeszukiwanie i optymalizacja - raport śródsemestralny}
\author{Maksym Bieńkowski, Jędrzej Grabski}
\date{15.01.2025}

\begin{document}
\maketitle
\begin{centering}
	\textbf{Temat projektu: }Algorytm roju cząstek z modyfikacjami dotyczącymi współczynnika bezwładności
\end{centering}

\section*{Postęp w projekcie}

W ramach prac nad projektem stworzono bazową wersję algorytmu PSO (Particle Swarm Optimization), w której współczynnik bezwładności nie jest jeszcze dynamicznie modyfikowany. Dodatkowo zaimplementowano następujące funkcjonalności:

\begin{itemize}
	\item \textbf{Wybór rodzaju zadania optymalizacji (enumeracja \texttt{Task})} \\
	      Umożliwienie definiowania różnych zadań, w których algorytm może być wykorzystany.

	\item \textbf{Rejestrowanie przebiegu optymalizacji (klasa \texttt{Logger})} \\
	      Gromadzenie danych o postępie i wynikach działania algorytmu w celu późniejszej, szczegółowej analizy.

	\item \textbf{Wizualizacja wyników (klasa \texttt{Plotter})} \\
	      Możliwość prezentacji zebranych danych w formie wykresów i analiz graficznych, co ułatwia wgląd w efektywność algorytmu.

	\item \textbf{Tryb pojedynczego uruchomienia algorytmu} \\
	      Jednorazowy start algorytmu, zebranie danych oraz ich wyświetlenie lub zapisanie w postaci obrazu, co usprawnia testy i demonstrację działania.
\end{itemize}

\section*{Dalsze prace}

W najbliższym czasie planowane są następujące kroki rozwojowe:

\begin{itemize}
	\item \textbf{Dynamiczna zmiana współczynnika bezwładności} \\
	      Wprowadzenie mechanizmu modyfikującego wartość współczynnika w trakcie działania algorytmu, aby lepiej dostosować się do różnych etapów optymalizacji.

	\item \textbf{Bardziej szczegółowe gromadzenie danych} \\
	      Rozszerzenie klasy \texttt{Logger} w celu zapisu dodatkowych statystyk.

	\item \textbf{Obliczanie wskaźników jakości} \\
	      Wprowadzenie omówionych w dokumentacji wstępnej miar oceny skuteczności algorytmu.

	\item \textbf{Wizualizacja wyników na przestrzeni wielu uruchomień algorytmu} \\
	      Rozbudowanie funkcjonalności klasy \texttt{Plotter} w taki sposób, aby możliwe było zestawienie rezultatów z kilku sesji optymalizacyjnych i ich porównanie.
\end{itemize}

\end{document}
